%%
%% Automatically generated file from DocOnce source
%% (https://github.com/hplgit/doconce/)
%%
%%


%-------------------- begin preamble ----------------------

\documentclass[%
oneside,                 % oneside: electronic viewing, twoside: printing
final,                   % draft: marks overfull hboxes, figures with paths
10pt]{article}

\listfiles               %  print all files needed to compile this document


\usepackage[totoc]{idxlayout}   % for index in the toc
\usepackage[nottoc]{tocbibind}  % for references/bibliography in the toc

\usepackage{relsize,makeidx,color,setspace,amsmath,amsfonts,amssymb}
\usepackage[table]{xcolor}
\usepackage{bm,ltablex,microtype}
\usepackage{comment} 
\usepackage[pdftex]{graphicx}

\usepackage{fancyvrb} % packages needed for verbatim environments

\usepackage[T1]{fontenc}
%\usepackage[latin1]{inputenc}
\usepackage{ucs}
\usepackage[utf8x]{inputenc}

\usepackage{lmodern}         % Latin Modern fonts derived from Computer Modern


\usepackage{pgfplotstable, booktabs}

\pgfplotstableset{
    every head row/.style={before row=\toprule,after row=\midrule},
    every last row/.style={after row=\bottomrule}
}





% Hyperlinks in PDF:
\definecolor{linkcolor}{rgb}{0,0,0.4}
\usepackage{hyperref}
\hypersetup{
    breaklinks=true,
    colorlinks=true,
    linkcolor=linkcolor,
    urlcolor=linkcolor,
    citecolor=black,
    filecolor=black,
    %filecolor=blue,
    pdfmenubar=true,
    pdftoolbar=true,
    bookmarksdepth=3   % Uncomment (and tweak) for PDF bookmarks with more levels than the TOC
    }
%\hyperbaseurl{}   % hyperlinks are relative to this root

\setcounter{tocdepth}{2}  % levels in table of contents

% --- fancyhdr package for fancy headers ---
\usepackage{fancyhdr}
\fancyhf{} % sets both header and footer to nothing
\renewcommand{\headrulewidth}{0pt}
\fancyfoot[LE,RO]{\thepage}
% Ensure copyright on titlepage (article style) and chapter pages (book style)
\fancypagestyle{plain}{
  \fancyhf{}
  \fancyfoot[C]{{\footnotesize \copyright\ 1999-2018, "Computational Physics I FYS3150/FYS4150":"http://www.uio.no/studier/emner/matnat/fys/FYS3150/index-eng.html". Released under CC Attribution-NonCommercial 4.0 license}}
%  \renewcommand{\footrulewidth}{0mm}
  \renewcommand{\headrulewidth}{0mm}
}
% Ensure copyright on titlepages with \thispagestyle{empty}
\fancypagestyle{empty}{
  \fancyhf{}
  \fancyfoot[C]{{ }}
  \renewcommand{\footrulewidth}{0mm}
  \renewcommand{\headrulewidth}{0mm}
}

\pagestyle{fancy}


% prevent orhpans and widows
\clubpenalty = 10000
\widowpenalty = 10000

% --- end of standard preamble for documents ---


% insert custom LaTeX commands...

\raggedbottom
\makeindex
\usepackage[totoc]{idxlayout}   % for index in the toc
\usepackage[nottoc]{tocbibind}  % for references/bibliography in the toc
\usepackage{listings}
\usepackage[normalem]{ulem} 	%for tables
\useunder{\uline}{\ul}{}
\usepackage{hyperref}
\usepackage[section]{placeins} %force figs in section

\usepackage{natbib}


%-------------------- end preamble ----------------------

\begin{document}

% matching end for #ifdef PREAMBLE

\newcommand{\exercisesection}[1]{\subsection*{#1}}


% ------------------- main content ----------------------



% ----------------- title -------------------------

\thispagestyle{empty}

\begin{center}
{\LARGE\bf
\begin{spacing}{1.25}
Problem set 1 - FYS3120
\end{spacing}
}
\end{center}

% ----------------- author(s) -------------------------

\begin{center}
{\bf Johan Nereng}
\end{center}

    \begin{center}
% List of all institutions:
\centerline{{\small Department of Physics, University of Oslo, Norway}}
\end{center}
    
% ----------------- end author(s) -------------------------

% --- begin date ---
\begin{center}
Des 14, 2018
\end{center}
% --- end date ---

\vspace{5cm}

\subsection{Problem 2}
Consider first a an Atwood machine with two parts; $m_1$ and $m_2$, moving vertically on a massless pulley, with a rope of fixed length, $l$, connecting the masses. Each mass has one degree of freedom (vertical movement along the y-axis), while the system set up imposes one constraint equation: $y_1+y_2=l$. Thus, a two body Atwood machine has $d=1$ degree of freedom. \par 

A compound Atwood machine of three masses, such as the one in fig.2 in the assignment paper, may be thought of as two separate two body Atwood machines, similar to the one described above. One Atwood machine with two parts; the part with mass $m_1$, and the part composed of the pulley to which $m_2$ and $m_3$ are connected, pulley 2, with mass $m_2+m_3$. The other Atwood machine is composed of the parts with $m_2$ and $m_3$. Each of the two Atwood machines has one degree of freedom, for a total of $d=2$ degrees of freedom. \par 
$d=2\implies q=2$ general coordinates: the vertical distances $Y_1$ and $Y_2$, where $Y_1$ is $y$ coordinate of pulley 2, and $Y_2$ is the distance from pulley 2 to the mass $m_2$. So;
\begin{align*}
y_1=Y_1-l_1 \\
y_3=Y_2-l_2\\
y_2=Y_2
\end{align*}
Thus
\begin{align*}
T=\frac{1}{2}m_1(\dot{Y_1})^2+\frac{1}{2}m_2(\dot{Y_2})^2+\frac{1}{2}m_3(\dot{Y_2})^2=\frac{1}{2}m_1(\dot{Y_1})^2+\frac{1}{2}(m_2+m_3)(\dot{Y_2})^2 \\
V=m_1g(Y_1-l_1 )+m_2g(Y_2-l_2)+m_3gY_2
\end{align*}
Such that $L=T-V=\frac{1}{2}m_1(\dot{Y_1})^2+\frac{1}{2}m_2(\dot{Y_2})^2+\frac{1}{2}m_3(\dot{Y_2})^2=\frac{1}{2}m_1(\dot{Y_1})^2+\frac{1}{2}(m_2+m_3)(\dot{Y_2})^2-m_1g(Y_1-l_1 )-m_2g(Y_2-l_2)-m_3gY_2$

\end{document}





% ------------------- end of main content ---------------



