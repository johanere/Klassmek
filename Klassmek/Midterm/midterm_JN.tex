\documentclass[11pt,a4paper]{report}

\usepackage[utf8]{inputenc}
\usepackage[T1]{fontenc}
\usepackage[english]{babel}
\usepackage{graphicx}
\usepackage{amsmath} % For \mathbb
\usepackage{amsfonts}
\usepackage{hyperref}

%\pagestyle{empty}


%% Numbered exercises
\newcounter{excount}[chapter]
\newenvironment{exercise}[1][]{\addtocounter{excount}{1} \noindent {\bf Question
    \arabic{excount} \ \ #1}\hspace{2mm}}{\vspace{4mm}}


\title{FYS3120 Classical mechanics and electrodynamics\\
Mid-term exam -- Spring term 2019}
\author{}



\begin{document}

\maketitle


\addtocounter{page}{1}

\section*{Important information:}
\begin{itemize}
\item
Your answers are to be submitted electronically as pdf-files, either generated from \LaTeX\  or scanned, at the latest Friday 29th of March at 16.00 local time (GMT+1). 
\item
This deadline is absolute.
\item
This mid-term exam counts for roughly  25\% of the total grade in FYS3120, and you must receive a passing score on the mid-term in order to pass the course.
\item
As this is a home-exam you are free to use any sources of information you may want, and you may collaborate with other students on solving the problems. However, the text of the submitted answers must be your own, and the usual rules of plagiarism apply. (We may check answers for similarities.)
\item
The best possible score on this exam is 25 points. Up to one point will be given for clear, concise and well presented answers, including appropriate figures and/or diagrams.
\item 
You may give your answers either in English or Norwegian.
\item
Good luck!
\end{itemize}

\cleardoublepage



%%%%%%%%%%%%%%%%%%%%
\begin{exercise}{\bf Central potentials\\}
%%%%%%%%%%%%%%%%%%%%
Consider two objects of mass $m_1$ and $m_2$ affected by a time-independent central potential $V(r)$ in regular three-dimensional space, where $\vec r$ gives the distance between the objects $\vec r = \vec r_1- \vec r_2$.

\begin{itemize}
\item[{\bf a)}] Find the number of degrees of freedom, and identify appropriate generalized coordinates. [1 point]
\item[{\bf b)}] Explain why the motion of three of the generalized coordinates can be solved trivially, and write down the solution. [1 point]
\item[{\bf c)}] Explain why the angular momentum of the reduced mass $\mu$,
\begin{equation}
\vec\ell = \vec r \times \vec p= \vec r \times \mu\dot{\vec r},
\end{equation}
where
\begin{equation}
\mu= \frac{m_1m_2}{m_1+m_2},
\end{equation}
is a constant of motion.
[1 point]
\item[{\bf d)}] Explain why this leads to the motion being confined to the plane perpendicular to $\vec\ell$. [1 point]
\item[{\bf e)}] Choosing $\vec\ell= \ell\hat k$, {\it i.e.}\ that the angular momentum is in the $z$-direction, we can switch to the polar coordinates $(r,\phi)$, where $\phi$ is the angle in the $xy$-plane. Show that the equations of motion for these coordinates are
\begin{equation}
\mu\ddot r-\frac{\ell^2}{\mu r^3}+\frac{\partial V}{\partial r}=0,\label{eq:LEQ_r}
\end{equation}
and
\begin{equation}
\dot\phi=\frac{\ell}{\mu r^2}\label{eq:LEQ_phi}.
\end{equation}
[3 points]
\item[{\bf f)}] Find the corresponding Hamiltonian in terms of the generalized coordinates $(r,\phi)$ and generalized momenta $(p_r,p_\phi)$, and write down Hamilton's equations for this system.  [3 points]
\item[{\bf g)}] Explain why the Hamiltonian $H$ of this system is a constant of motion and what the physical interpretation of $H$ is. [1 point]
\end{itemize}
\end{exercise}


%%%%%%%%%%%%%%%%%%%%
\begin{exercise}{\bf Orbital motion\\}
%%%%%%%%%%%%%%%%%%%%
Consider the gravitational potential between two objects of mass $m_1$ and $m_2$,
\begin{equation}
V(r)=-\frac{Gm_1m_2}{r},\label{eq:grav_pot}
\end{equation}
where $G$ is Newton's gravitational constant.
\begin{itemize}
\item[{\bf a)}] Find Hamilton's equations for this system. [1 point]
\item $K=\frac{1}{2}m_1v_1^2+\frac{1}{2}m_2v_2^2$
\item $V=V(r)=-\frac{Gm_1m_2}{r}$
\item $H=K+V=\frac{1}{2}m_1v_1^2+\frac{1}{2}m_2v_2^2-\frac{Gm_1m_2}{r}$
\item $\dot{r}_1=v_1$

\item -----
\item Using coordinates $q_1=r$ and $q_2=R$, such that $\vec{r}=\vec{r_1}-\vec{r_2}$ and $\vec{R}=\frac{\mu}{m_2} \vec{r_1}+\frac{\mu}{m_1}\vec{r_2}$, where $\mu=\frac{m_1m_2}{m_1+m_2}$ is the reduced mass. This yields the Lagrangian (\textbf{vise?})
\item $L=\frac{1}{2}(m_1+m_2)\dot{\vec{R}}^2+\frac{1}{2}\mu\dot{\vec{r}}^2-V(r)$. 
\item Finding the momenta (\textbf{finne direkte i stedet md K+V?)}
\begin{align}
&p_i=\frac{\partial L}{\partial \dot{q_i}}\\
&p_1=\frac{\partial L}{\partial \dot{r}}=\frac{\partial }{\partial \dot{r}}  \left( \frac{1}{2}(m_1+m_2)\dot{\vec{R}}^2+\frac{1}{2}\mu\dot{\vec{r}}^2+\frac{Gm_1m_2}{r}\right)=\mu \dot{\vec{r}} \\
&p_2=\frac{\partial L}{\partial \dot{R}}=(m_1+m_2)\dot{\vec{R}}
\end{align} 
\item Finding the Hamiltonian: $H=\sum_ip_i\dot{q_i}-L=\mu \dot{\vec{r}}^2 +(m_1+m_2)\dot{\vec{R}}^2-\frac{1}{2}(m_1+m_2)\dot{\vec{R}}^2-\frac{1}{2}\mu\dot{\vec{r}}^2-\frac{Gm_1m_2}{r}=\frac{\vec{p_1}^2}{2\mu}+\frac{\vec{p_2}^2}{2(m_1+m_2)}-\frac{Gm_1m_2}{r}$
\item Finding the Hamilton's equations:
\begin{align*}
\dot{q_i}=\frac{\partial H}{\partial p_i}

\dot{p_i}=-\frac{\partial H}{\partial q_i}$
\end{align*}

 \item[{\bf b)}] Make a two-dimensional plot of the phase space for the distance $r$ between the objects, and its generalized momentum $p_r$, for concreteness fixing the masses to be those of the International Space Station (ISS) orbiting the Earth, varying $\ell$.

Give a qualitative description of the different types of motion that can be read out of the diagram and comment on how the situation changes with increasing $\ell$. {\it Hint:} For those using {\tt python} the plotting framework {\tt pyplot} has a useful plotting command for phase space plots called {\tt streamplot}.  [4 points]
\item[{\bf c)}] Use the Lagrange equations to show that the differential equation for the orbit equation $r(\phi)$, where $\phi$ is the polar angle of the motion, is
\begin{equation}
\frac{d^2r}{d\phi^2}-\frac{2}{ r}\left(\frac{dr}{d\phi}\right)^2-r+\frac{Gm_1m_2\mu }{\ell^2}r^2=0.
\end{equation}
[2 points]
\item[{\bf d)}] Show that complete set of solutions for the orbit equation  is
\begin{equation}
r(\phi)= \frac{r_0}{1-\varepsilon \cos(\phi-\phi_0)},
\end{equation}
where $\varepsilon$ is called the {\bf eccentricity} and $\phi_0$ the {\bf phase}. Find $r_0$.
{\it Hint:} The substitution $s=1/r$ is very useful. [2 points]
\item[{\bf e)}] Find an expression for the total energy in terms of the eccentricity $\varepsilon$. [2 points]
\item[{\bf f)}] If an astronaut jumped off the ISS directly towards Earth, what would happen to her orbit? Assume, for simplicity, that the ISS is in a circular orbit. [2 points]
\end{itemize}
\end{exercise}



\end{document}